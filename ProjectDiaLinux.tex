\documentclass[11pt]{article}
\usepackage{graphicx}
\newcommand{\HRule}{\rule{\linewidth}{0.5mm}}
\begin{document}
\begin{titlepage}
 
\begin{center}
 

% Upper part of the page
\begin{figure}
\includegraphics[width=4in]{diatitel.png}  
\caption [Logo]{Logo van de applicatie.}
\end{figure}   
 
\textsc{\LARGE \bf Project 2TIN 2011-2012}\\[1.5cm]
 
\textsc{\Large Linux-Unix}\\[0.5cm]
 

% Title
 \HRule \\[0.4cm]
 { \huge \bf Dia}
 
\HRule \\[1.5cm]
 
% Author

 \emph{Author:}\\
 Jonas \textsc{Tielens}
 

 
\vfill
 
% Bottom of the page
 {\large \today}
 
\end{center}
 
\end{titlepage}

\newpage
\tableofcontents
\newpage

\section {Inleiding}
In de richting 2TIN aan de Provinciale Hogeschool Limburg, is een vak genaamd: “Linux-Unix”. Voor dit vak hebben wij de opdracht gekregen om een Linux-app voor te stellen. Dit alles moeten we verwezenlijken via LaTeX. Ik ga dit doen voor de app “Dia”. Wat deze app juist is en hoe je het moet gebruiken, dat kom je later in de tekst te weten.

\section {Wat is Dia?}
Dia is een software programma waarmee je diagrammen kan tekenen. Het is te vergelijken met het programma van Microsoft, namelijk “Visio”. Dia kan je ook installeren onder Windows, GNU/Linux, Unix en MacOS X. Maar Het wordt voornamelijk gebruikt met Linux of Unix. Het wordt het meeste gebruikt voor het maken van UML-Diagrammen, stroomdiagrammen... . Maar er zijn uiteraard ook nog andere diagrammen die je kunt maken met Dia. Je kunt ook je Diagrammen exporteren naar bepaalde formaten, of je kan ze afdrukken. Dia heeft veel functies geïmplementeerd dus ja kan er vlot mee werken.

\begin{figure}[hb]
\centering
\includegraphics[width=4in]{voorbeeldHoofdschermDia.png}  
\caption [Hoofd Scherm]{Dit is een voorbeeld van het hoofdscherm waarop een diagram gemaakt is.}
\end{figure}   

\section {Hoe te installeren?}

\section {Verschil tussen Dia en MS Visio.}

\section {Voor- en nadelen van Dia.}
\subsection{Voordelen}
hier komen de voordelen
\subsection{Nadelen}

\section{Installatie en configuratie van subversionsysteem.}

\section{Mail configureren onder Ubuntu.}
\end{document}
